% !TeX spellcheck = en_US
% Chapter 2
\chapter{مرور}\label{chap:review}

\section{بخش اول}\label{chap:brain}

این یک متن نمونه است...این یک متن نمونه است...این یک متن نمونه است...این یک متن نمونه است...این یک متن نمونه است... این یک متن نمونه است...این یک متن نمونه است...
برای بیان اکرونیم یک کلمه مانند  \gls{LASSO}، به این صورت عمل کنید.
   \gls{ML}
   یا 
   \gls{DL}
   یا
   \gls{FN}
   یا 
   \gls{TN}
   یا
   \gls{LSTM}
   یا 
   \gls{SGL}
   یا
   \gls{TP}
   یا 
   \gls{FP}
   یا
\subsection{زیربخش \gls*{1st}}\label{sec:brain:sub_brain}
این یک متن نمونه است که به سه مرحله \gls{1st}،
 \gls{2nd} و \gls{3rd} تقسیم می‌شود. 
 
 برای معادل جمع فارسی در دیکشنری اینطور فراخوان کنید: \glspl{2nd}
\begin{itemize}
	\item
	\gls{1st}: 
	این یک آیتم نمونه است...
		\item 
	\gls{2nd}: 
	این یک آیتم نمونه است...
	\item 
	\gls{3rd}: 
	این یک آیتم نمونه است...
\end{itemize}

\subsection{زیربخش دوم}\label{sec:brain:second_sub}
این یک متن نمونه است...


\paragraph{پاراگراف نمونه \gls*{1st}}
این یک متن نمونه است...
\paragraph{پاراگراف نمونه \gls*{2nd}}
این یک متن نمونه است...

\section{معرفی \gls*{dataset}}\label{sec:dataset}
این بخش نمونه است...

\begin{enumerate}
	
\item\textbf{فاز اول یا \lr{English-1}:} 
این یک متن نمونه است.
\begin{itemize}
	\item 
	این یک آیتم نمونه است...
	\item 
	این یک آیتم نمونه است...
	\item 
	این یک آیتم نمونه است...
\end{itemize}

این یک متن نمونه است...

\item\textbf{فاز دوم یا \lr{English-2}:} 
این یک متن نمونه است...

\item\textbf{فاز سوم یا \lr{English-3}:} 
این یک متن نمونه است...
\end{enumerate}

برای کامنت کردن بخشی از پروژه بدون حذف آن، از این دستور استفاده کنید.
\begin{comment}
برای کامنت کردن بخشی از پروژه بدون حذف آن، از این دستور استفاده کنید.
\end{comment}

برای درج جدول \ref{tab:my_table1} و رفرنس‌دهی از این کد استفاده کنید.

\begin{table}[t!]
	\centering
	\caption{جدول نمونه افقی}
	\label{tab:my_table1}
	\begin{tabular}{rccccc}
		\toprule			
		$\text{گروه}$ && $\text{فارسی (English)}$ & $\text{فارسی English}$ & $\text{فارسی English}$ & $\text{فارسی English}$ \\
		\midrule
		\rl{جنسیت}&(\rl{مرد - زن})	
		& $10 - 20$ & $10 - 20$ & $10 - 20$ & $10 - 20$
		\\
		\bottomrule
	\end{tabular}
\end{table}

برای درج جدول عمودی \ref{tab:my_table2} و رفرنس‌دهی از این کد استفاده کنید.
\begin{sidewaystable}
	\begin{threeparttable}
		\caption{جدول نمونه عمودی}
		\label{tab:my_table2}
		\setlength\tabcolsep{0pt} % make LaTeX figure out intercolumn spacing
		
		\begin{tabular*}{\columnwidth}{@{\extracolsep{\fill}} cccccccccc}
			\toprule			
			$\text{گروه}$ & 
			$\myfrac{\text{جنسیت}}{\text{مرد - زن}}$ & 
			$\myfrac{\text{فارسی}}{\text{\lr{mean$\pm $std}}}$ & 
			$\myfrac{\text{ٍenglish}}{\text{\lr{mean$\pm $std}}}$ & 
			$\myfrac{\text{فارسی}}{\text{\lr{mean$\pm $std}}}$ & 
			$\myfrac{\text{\lr{english}}}{\text{\lr{mean$\pm $std}}}$ & 
			$\myfrac{\text{\lr{ٍenglish}}}{\text{\lr{mean$\pm $std}}}$ & 
			$\myfrac{\text{ٍenglish}}{\text{\lr{mean$\pm $std}}}$ & 
			$\myfrac{\text{\lr{ٍenglish}}}{\text{\lr{mean$\pm $std}}}$ & 
			$\myfrac{\text{\lr{RAVALT-PF}}}{\text{\lr{mean$\pm $std}}}$ \\
			\midrule
			فارسی AD & $189 / 153$ & $75.0 \pm 7.8$ & $4.39 \pm 1.67$ & $15.2 \pm 3.0$ & $19.6 \pm 6.7$ & $29.9 \pm 8.1$ & $23.2 \pm 2$ & $22.8 \pm 7.6$ & $89.0 \pm 21.2$ 
			\\
			\bottomrule
		\end{tabular*}
	\end{threeparttable}
\end{sidewaystable}

\section{فرمول}\label{sec:formula}
بنابراین برای هر نقطه $X$ از تصویر سه بعدی دریافتی، رابطه
\begin{align} \label{eq:n3-main}
	v(X) = u(X) f(X) + n(X)
\end{align}
برقرار است که $v$ سیگنال اندازه‌گیری شده، $u$ سیگنال مورد انتظار بدون اثر \gls{bias} میدان و $f$ یک سیگنال با تغییرات آرام و نماینده \gls{bias} میدان است.
روابط تبدیل فوریه تابع $f(t)$ و معکوس آن به بیان شده در روابط \ref{eq:fourier} و \ref{eq:afourier}  هستند.
\begin{align}\label{eq:fourier}
\mathcal{F}(f(t))= F(w)=\sum_{k=0}^{N-1}f_k e^{-i\frac{2\pi}{N}kn}
\end{align}
\begin{align}\label{eq:afourier}
f_k= F(w)=\frac{1}{N}\sum_{n=0}^{N-1}\mathcal{F}_n e^{-i\frac{2\pi}{N}kn}
\end{align}

